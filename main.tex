\documentclass{article}

\usepackage[english]{babel}

% Set page size and margins
\usepackage[letterpaper,top=2cm,bottom=2cm,left=3cm,right=3cm,marginparwidth=1.75cm]{geometry}

\usepackage{amsmath}
\usepackage{graphicx}
\usepackage{float}

\title{The Cheetah (Acinonyx jubatus): Speed, Adaptations, and Conservation}
\author{Syed Muhammad Taha - 24K-0706}

\begin{document}
\maketitle

\begin{abstract}
The cheetah (\textit{Acinonyx jubatus}) is renowned as the fastest land animal, capable of reaching speeds up to 70 miles per hour. This large cat species is highly specialized for sprinting, with unique adaptations that make it an exceptional hunter in open grasslands. However, despite its agility and speed, the cheetah faces numerous threats that have led to a decline in its populations across Africa and parts of Iran. This paper examines the cheetah’s anatomical adaptations, ecological role, and the current conservation efforts necessary to protect it. We also propose a hypothetical formula to estimate the lifespan of cheetahs based on ecological and physiological variables.
\end{abstract}

\section{Introduction}

The cheetah (\textit{Acinonyx jubatus}) is a unique member of the cat family (Felidae) and is distinct for its extraordinary speed and sleek body structure, which make it the fastest terrestrial animal. Native to Africa and some parts of Iran, cheetahs prefer open savannas and grasslands where they can use their speed to hunt antelope and other fast-moving prey. However, habitat loss, human-wildlife conflict, and illegal wildlife trade have placed the species at significant risk. In this paper, we explore the cheetah’s biological characteristics, behavior, and the conservation challenges it faces.

\begin{figure}[H]
    \centering
    \includegraphics[width=0.6\textwidth]{cheetah_running.jpg}
    \caption{Cheetah Running}
\end{figure}

\section{Physical and Behavioral Adaptations}

\subsection{Physical Characteristics}

The cheetah is highly adapted for running, with long, slim legs, a small, rounded head on a long neck, a deep chest, and special paw pads for traction. Unlike most big cats, the cheetah’s claws are only partially retractable, providing extra grip during high-speed chases. Cheetahs also have an elongated spine and flexible hip joints, allowing for a greater stride length and enhanced flexibility during sprints.

\subsection{Hunting and Feeding Behavior}
Cheetahs are diurnal hunters, primarily active during the early morning and late afternoon. Their hunting strategy relies on stealth, a rapid burst of speed to close in on prey, and precise timing to avoid energy expenditure. Cheetahs typically target medium-sized ungulates, such as gazelles and impalas, which they catch by accelerating from 0 to 60 miles per hour in just a few seconds.

\begin{figure}[H]
    \centering
    \includegraphics[width=0.6\textwidth]{cheetah_hunt.jpg}
    \caption{Cheetah hunting a deer}
\end{figure}

\section{Ecological Role and Conservation Status}

\subsection{Role in Ecosystem}
As apex predators, cheetahs play an important role in maintaining the health of prey populations by targeting the weakest individuals, thus promoting genetic fitness among herbivores. Their presence also impacts other predators, as cheetahs often lose their kills to larger animals like lions and hyenas, affecting inter-species dynamics within ecosystems.

\begin{figure}[H]
    \centering
    \includegraphics[width=0.6\textwidth]{cheetah-eating-prey.jpg}
    \caption{Cheetah eating its prey}
\end{figure}

\subsection{Conservation Challenges}
Today, cheetahs are listed as vulnerable, with populations declining due to habitat loss, human-wildlife conflict, and illegal poaching. Conservation organizations and governments are working together to establish protected areas, anti-poaching patrols, and community education programs to help mitigate these threats.

\begin{table}[H]
    \centering
    \begin{tabular}{|l|l|}
        \hline
         \textbf{Characteristic} & \textbf{Details} \\
         \hline
         Scientific Name & \textit{Acinonyx jubatus} \\
         \hline
         Class & Mammalia \\
         \hline
         Diet & Carnivore; primarily feeds on medium-sized ungulates such as gazelles and impalas \\
         \hline
         Average Lifespan & 10-12 years in the wild; up to 17 years in captivity \\
         \hline
    \end{tabular}
    \caption{Summary of Cheetah Biological Classification and Characteristics}
\end{table}

\section{Hypothesis about Cheetah Lifespan}
To hypothesize about the factors that may influence a cheetah's lifespan, we propose a formula that considers variables such as metabolic rate, environmental stressors, and access to prey. This hypothetical formula might look as follows:

\begin{figure}[H]
    \centering
    \includegraphics[width=0.6\textwidth]{cheetah_life.jpg}
    \caption{Female cheetah with its cubs}
\end{figure}

\begin{equation}
    \text{Lifespan} = \frac{k \times (1+P)}{E+M}
\end{equation}
Where:

\(k\) is a constant representing baseline lifespan in optimal conditions.  
\(P\) is a factor representing prey availability (e.g., relative abundance).  
\(E\) represents environmental stress factors (e.g., human impact, habitat loss).  
\(M\) represents metabolic rate, adjusted for activity levels associated with hunting.

This formula suggests that a cheetah’s lifespan could be extended with higher prey availability (higher \(P\)) and reduced environmental stress (lower \(E\)). The metabolic rate factor \(M\) accounts for the high energy demands of cheetahs due to their frequent high-speed chases, which may reduce lifespan under stress.

\section{Conclusion}
The cheetah remains a marvel of natural evolution, balancing incredible speed with specialized adaptations that allow it to survive in a competitive environment. However, the cheetah's existence is threatened by a rapidly changing world. Protecting cheetah habitats, addressing human-wildlife conflict, and enforcing anti-poaching measures are crucial to ensuring the survival of this species. Future research and conservation efforts will be vital in understanding and mitigating the challenges faced by this unique predator.

\section{References}
\bibliographystyle{plain}
\bibliography{references}


\end{document}
